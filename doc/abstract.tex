%******  Abstract **************
\begin{abstract}
The major objective of this study is to explore the possibility of combining a traditional rigid body simulation with a deformable body simulation. The deformable bodies are implemented according to the relatively new approach of using oriented particles. The combined bodies consist of a rigid inner structure and a soft outer structure. The goal is to improve upon traditional system simulating skeletons. The bones can now be modeled with a surrounding cushion of tissue using the presented method. This extends and improves simpler models using only rigid bodies especially in areas where the number of contact points is important to achieve a stable simulation.  Using the soft deformable tissue enables bodies to surround other bodies more realistically while increasing the number of contact points significantly. Thus this combined simulation model thus can provide a more stable simulation.

The main contribution of this thesis is the modeling and description of the interface between the deformable oriented particles and rigid bodies. The first interface is inside the body itself, between the bones and the tissue which requires an efficient way to closely couple both subsimulations. The second interface is between these combined bodies and rigid bodies in the simulated world. Here the impulses generated by the tissue onto the rigid bodies has to be defined and calculated.

The outcome shows that the simulation provides realistic results. For the simple scenario of picking up a cylinder only two rigid bodies, modeling the fingers, are needed. A rigid body only simulation would either completely fail at this task or would require multiple bodies to simulate the whole finger. This result promises both more efficient and faster simulations of complete skeleton models.
\end{abstract}
