\chapter{Introduction}
\label{cha:introduction}

Simulating the physically properties of objects with the help of a computer is a very important topic across a wide range of fields. As computing power is both expensive and limit the approach to simulations is distributed across a spectrum of accuracy at the one end and performance on the other. Some fields require the best and most accurate simulations currently available. These are especially predominant in applications where predictions for the real world behavior of the system are required and later applied to constructing and optimizing real world objects. Example fields for this are automotive or aerospace modeling and simulations. Here extremely detailed models are used to predict the properties and behavior of all kinds of different materials, shapes and bodies. The results of these simulations directly influence the construction and design decisions for their real world counter parts. Simulating the objects first inside the computer enables a faster development cycle and thus provides an efficient and cost effective way of constructing new parts and pieces.

Other fields are not reliant on the direct applicability of their results to the real world. The main focus of these simulations is to provide physically plausible and visually believably results in a timely fashion. Probably the two most popular and widely known application for these types of simulations are Games and Animations. Here the goal is to introduce the viewer into a completely virtual universe where everything is controlled by the media creators. In order to achieve the desired immersion of the viewers realistic behavior of the actors and objects in this universe is really important. Obvious incorrect physical behavior is most of the time immediately recognizable by the consumers and can easily break their immersion in the content.

Today games and animations use a huge number of interacting bodies. These bodies are simulated using a variety of different simulation methods and techniques. For animating characters and objects in movies the main focus lies in reducing the workload of animators, who traditionally had to animate and model everything by hand, a long and tedious process. Physical simulations can aid the animators in automatically generating physically plausible animations to start with. These animations can then be later adjusted and modified as necessary but a huge amount of work can be done by the simulation. These techniques and methods are only secondarily focused on the simulation time, while this is still a consideration to enable faster turn around times for the animator playing with different ideas the primary focus of these systems is the controllability of the output. These system should enable the animators to define their desired behavior and letting the system automatically figure out how to simulate everything in between, always trying to have an at least physical plausible believable result.

Physical simulations in games on the other hand have to primarily focus on the real time aspect of their respective techniques and methods. Today games simulate a huge a amount of objects in an ever increasing world. All these simulations have to calculated in an extremely short period of time. The simulated physics have to share the resources with everything else required to simulate the game world including game logic, artificial intelligence and computer graphics. In order to achieve smooth 60 fps all these different subsystems have to be extremely efficient and fast. If the simulations slow down too much the immersion of the player is immediately broken when things start to move unrealistically.

Another interesting subarea of the games are the so called serious games. Here the focus of the simulated world is not to provide fun and entertainment to the player. Instead the goal is the model the real world as realistically as possible to enable training and orientation for people working in the simulated field of expertise. Prominent examples are military and emergency personnel. Simulations can be used to give these persons a chance to experience real world scenarios in a controlled environment so they can train and prepare themselves for their duties. These types of games are also used to help train medial personnel to perform surgeries on completely simulated patients first before applying the knowledge to real world cases. Providing realistic real time feedback to the users is the main focus of these types of simulations.

For all close to real time and controllable simulations rigid body simulations have been a very popular choice. Rigid body simulations enable fast and efficient simulations of a huge number of interacting bodies. This can be achieved with the assumption that all bodies in the simulation a completely rigid and can never deform. This assumption is plausible for a wide variety of applications ranging from simple car models to more complex ragdoll models. Unfortunately not all possible scenarios can be modeled with simplified rigid bodies. Objects in the real world a very often deformable and bend and stretch under the influence of external forces. Incorporating deformable simulations models into games and animations enable a whole range of new interesting scenarios. However the simulation of deformable bodies requires a lot more computing resources to achieve realistic results. Novel ways to reduce the amount of work associated with deformable simulations of the be found.

One such approach is to combine the two different simulation approaches. Here the goal is combine the very fast rigid body simulations with the slower deformable simulations in order to reap the benefits of both simultaneously. Some approaches try to simulate rigid bodies and deformable bodies side by side for each object, while other approaches try to model a single body using both rigid as well as deformable parts. A real world example of the second time is the human body itself. Here basically rigid bodies, the bones, form an underlying rigid skeleton. However in contrast to simple ragdolls the reality this rigid skeleton is surrounded by a soft layer of tissue. This soft tissue can deform when forces are applied. Modeling this seemingly obvious scenario efficiently is non trivial and a perfect match for a combined simulation model.

The focus of this thesis is to explore this type of simulation, where a rigid inner core is simulated with a soft outer layer of deformable material. The goal is have an efficient simulation with is both fast and robust in scenarios where only using a simplified rigid body simulation would give unsatisfactory results. The thesis furthermore focusses on a specific technique to implement the deformable material. This technique uses oriented particles to simulate the deformations of a body. Oriented particle can be simulated efficiently and in realtime while still giving physically plausible results.

The thesis first gives a short background on how rigid body simulations are implemented and how the simulation is used in the proposed method. After that a more in-depth description of the way oriented particles are implemented is presented, beginning with position based dynamics and shape matching and explaining how these approaches enable oriented particles. The final part describes the method and ideas for combining these two different approaches into one system.