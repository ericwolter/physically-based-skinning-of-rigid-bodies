\chapter{Related Approaches}
\label{cha:related_approaches}

The approach described in this thesis uses oriented particles to simulate soft tissue around a rigid inner bone. The combination of rigid body simulations with deformable body simulation have been explorer very early by Terzopoulos et al. \cite{Terzopoulos:1988bz} shortly after deformable bodies where first introduced in \cite{Terzopoulos:1987gf}. Since then many different techniques and methods have been researched. A recent review of different approaches can be found in \cite{Nealen:2006vp}. The work in this thesis builds upon particle based simulations and in particular the approach by M{\"u}ller et al. \cite{Muller:2011gn}. The approach itself combines and extents the two previous methods of shape matching described in \cite{Muller:2005fi} and position based dynamics found in \cite{Muller:2007vs} by additional including the orientation of each particle into the calculations.

Their are two different ways of approaching the combination of rigid body simulations and deformable body simulations. The first focusses on a two-way coupling of the bodies. This mean each body is simulated using its own simulation. Both types are then allowed to interact in the same simulation world. For example Jansson et al. \cite{Jansson:2003cb} used a mass-spring representation extended by the concept of volume to simulate both rigid bodies and deformable bodies. Thus they had no need to explicitly handle the interaction as both types of bodies used the same underlying representation.

Lenoir et al. \cite{Lenoir:2004ic} explored a more generic constraint solver using Lagrange theory to simulate deformable models and a classical Newton-Euler formalism to simulate rigid bodies. In 2007 both \cite{Muller:2007vs} and \cite{Sifakis:2007to} extended their respective frameworks to handle two-way coupling between rigid and deformable bodies. Another improvement followed in \cite{Shinar:2008va} which main focus was on the stability of coupling. They achieved this by using a unified time integration for both simulations as well as an individual two-way coupling algorithm for each body type.

The second approach for combining rigid bodies with deformable bodies focusses on modeling an inner skeleton to drive the outer deformable parts. One of the first such ideas was explored in \cite{Capell:2002ge}. They utilized the finite element method (FEM) to simulate the deformable body but in addition introduced line constraints along the bones of the skeleton. Later Georgii et al. picked up on this general idea in \ref{Georgii:2010tw} by including a more advanced two-way coupling algorithm based on a geometric multigrid scheme to solve the FEM equations governing the deformable body parts.

In the paper \cite{Jain:2011hb} Jain et al. explore a very similar approach to the one taken in this thesis. They also use articulated rigid bones and surround them with a set of point masses that simulate the deformable body parts. Improving the performance of these types of simulations with an inner skeleton is the main focus of \cite{Kim:2011iq}. Instead of point based deformable bodies they choose to implement non linear finite elements in order to negate the problems arising from the linearized stress when the deformations contain rotational modes. A problem which is also addressed by incorporating the orientation information into the deformable model as done by the oriented particles approach in \cite{Muller:2011gn}. 

Another interesting paper by Stavness et al. \cite{Stavness:2011fn} explores the application of skeleton based deformable models to simulate real world body parts for medical applications. Including a FEM model for the tongue, muscle actuators, constraints for the bite contact and rigid bones for the jaw.

Applying rigid skeletons combined with soft deformable outer layers to the field of computer animation is the focus of a paper by McAdams et al. \cite{McAdams:2011ht}. They utilize a hexahedral lattice to simulate the deformable skin of the animated character which is solved using a multigrid method. M{\"u}ller et al. themselves extend upon their original oriented particles approach in \cite{Muller:2011vu}. They present a method to apply oriented particles to enhance the realism of animated characters. However, instead of explicitly modeling the bones as rigid bones they link the particles to a regular mesh representing the underlying skeleton structure.