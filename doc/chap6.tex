\chapter{Solution Details}
\label{cha:solution_details}

\begin{itemize}
\item Implementation combines rigid body simulation and oriented particles
\item Using bullet library for the managing the rigid body state
\item Includes forces, integrating velocities, and predicting unconstrained motion
\item Bullet is also used for applying impulses generated by the tissue and finally updating the rigid bodies state
\item Oriented particles implementation is completely custom
\end{itemize}

\section{Setup}
\begin{itemize}
\item Settings up the environment
\item In total four bodies are simulated
\item ground, 2 fingers, cylinder
\item ground is a simple static rigid body and does not move are react to force during the simulation
\item fingers and cylinder are all combined bodies
\item Setting up a combined body
\item Each inner rigid body has a arbitrarily chosen mass
\item The underlying geometric structure is used to evenly distribute attached particles across its surface
\item The density of the attached particles can be controlled by a configurable $density$parameter
\item For each attached particle on the surface a corresponding outer particle is generated
\item The outer particle is positioned along the same vector as its corresponding attached particle relative to the center of mass of the rigid inner core
\item The distance between attached and outer particle can be controlled by a configurable $extrude$ parameter
\item After all the particles have been added to the surrounding tissue the implicit shape matching groups are generated
\item The size of each group is a configurable $groupSize$ parameter
\item For each particle the $n = groupSize - 1$ closest particles using simple euclidean distance are added to an implicit group with this particle at the center
\item In addition to the closest particles the corresponding attached or outer particle is also added to the group
\item Thus the group size for each particle becomes exactly $groupSize$
\end{itemize}
\section{Loop}
\begin{itemize}
\item For all active bodies gravity is applied and the velocities updated
\item After that an unconstrained motion is predicted
\item Both the rigid body is temporarily moved to a new position as well as all the particles belonging to the body
\item This predicts both the positions and orientations
\item Predicted position of particles are adjusted in case of potential collisions
\item Gauss-Seidel type iterations are performed a configurable number of times in order to resolve the shape matching constraints
\item After arriving at the final shape matched configuration of predicted position and orientation for the particles, the error between the fixed position of attached particle and their predicted position is evaluated
\item For each error an impulse is applied to the predicted transform of the rigid body
\item After applying the impulses the predicted position and orientation of the attached particles is forcefully set to their fixed position and orientation relative to position of the rigid body
\item In the final step the the position/orientation of the rigid body and all particles is updated to their new values
\end{itemize}